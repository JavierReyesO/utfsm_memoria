\section{Propuesta de Solución}

La solución propuesta considera dos etapas. En primer lugar se realiza una detección de características finas en la malla de superficie entrante. Luego, con la información proporcionada se procederá a corregir la representación en la malla de volumen.

\subsection{Detección de características}

Para la detección de características se 

%\begin{algorithm}
%\caption{Detección de Características v1.0}
%\label{alg:deteccion2}
%\begin{algorithmic}[1]
%\REQUIRE $Q:$ Conjunto de vértices y $E:$ Conjunto de arcos
%\STATE $Q \leftarrow V \setminus\{v\}$
%\STATE $l[v] \leftarrow 0$
%\STATE Marque todos los vértices $x \in Q$ con $l[x] = \infty$
%\FOR{$i \leftarrow 1$ \TO $|V|-1$}
%\FORALL{$xy \in E$}
%\STATE $l[y] \leftarrow \min{\{l[y],l[x]+w(xy)\}}$
%\ENDFOR
%\ENDFOR
%\IF{hay un arco $xy$ con $l[x]+w(xy)<l[y]$}
%\STATE \COMMENT{Hay un ciclo de costo negativo}
%\ENDIF
%\STATE $n \leftarrow \{\}$
%\FORALL{$e$ in elementos}
%\STATE $n[$\texttt{index($e$)}$] \leftarrow $\texttt{calcularnormal($e$)}
%\ENDFOR
%\FORALL{$e$ in elementos}
%\STATE $contador \leftarrow 0$
%\STATE $ne \leftarrow n[$\texttt{index($e$)}$]$
%\STATE $v \leftarrow $\texttt{vecinos($e$)}
%\FORALL{$vecino$ in $v$}
%\STATE $nvec \leftarrow n[$\texttt{index($vecino$)}$]$
%\STATE $\theta \leftarrow $\texttt{arccos(productopunto($nvec,ne$)/(norma($nvec$)$\cdot $norma($ne$)))}
%\IF{$\theta > \theta_s$}
%\STATE $contador++$
%\ENDIF
%\ENDFOR
%\ENDFOR
%\end{algorithmic}
%\end{algorithm}

\begin{algorithm}
\caption{Detección de Características}
\label{alg:deteccion}
\begin{algorithmic}[1]
\REQUIRE $E:$ Conjunto de arcos $F:$ Conjunto de Caras
\STATE $E' \leftarrow \emptyset$
\FORALL{$e$ $\in$ $E$}
\STATE $f_1$, $f_2$ $\leftarrow$ \texttt{obtenerCarasArcoComun($e$,$F$)}
\STATE $n_1 \leftarrow$ \texttt{obtenerVectorNormal($f_1$)} 
\STATE $n_2 \leftarrow$ \texttt{obtenerVectorNormal($f_2$)}
\STATE $\theta \leftarrow $ \texttt{anguloEntreNormales($n_1,n_2$)}
\IF{$\theta > \theta_s$}
\STATE $E' \leftarrow E' \cup e$
\ENDIF
\ENDFOR
\end{algorithmic}
\end{algorithm}

\subsection{Representación de características}

soon... 
