\section{Descripción del Problema}

La mallas de volumen son usadas en una variedad de problemas de simulaciones geométricas. Estas mallas son generadas por el Método de Elementos Finitos (FEM, por sus siglas en inglés, \textit{Finite Element Method}) y típicamente se utilizan hexahedros para representar un dominio geométrico discretizado. El proceso para la generación de mallas es el método Octree. Este método recibe como entrada una malla de superficies, que es una representación que considera solo el exterior de una geometría y generar una malla de volumen, en donde también es posible apreciar el interior.\\

Al representar mallas de volumen se consideran distintos tipos de geometría. Hay geometrías simples, complejas, curvas y/o rectas. Ejemplos de generación de mallas existen con fines médicos, cuando se desea generar la malla de volumen un fémur \cite{Lobos2013a}, o con fines científicos, donde se puede encontrar generación de mallas de volumen de partes mecánicas complejas \cite{Marechal2009}.\\

Describir Problema!!!


\section{Objetivos}

\begin{enumerate}
\item Objetivo General
\begin{itemize}
\item Detectar características finas en mallas de superficie y considerarlas en algoritmos de generación de mallas de volumen.
\end{itemize}
\item Objetivos Específicos:
\begin{itemize}
\item Investigar y revisar técnicas utilizadas en la detección de características en ámbitos relacionados.
\item Establecer una metodología para la detección de características en mallas de superficie.
\item Implementar un algoritmo que considere las características detectadas y mejore la representación de las mallas de volumen.
\item Comparar la solución escogida con la implementación anterior.
\end{itemize}
\end{enumerate}
